\PassOptionsToPackage{unicode=true}{hyperref} % options for packages loaded elsewhere
\PassOptionsToPackage{hyphens}{url}
%
\documentclass[12pt,english,]{article}
\usepackage{lmodern}
\usepackage{amssymb,amsmath}
\usepackage{ifxetex,ifluatex}
\usepackage{fixltx2e} % provides \textsubscript
\ifnum 0\ifxetex 1\fi\ifluatex 1\fi=0 % if pdftex
  \usepackage[T1]{fontenc}
  \usepackage[utf8]{inputenc}
  \usepackage{textcomp} % provides euro and other symbols
\else % if luatex or xelatex
  \usepackage{unicode-math}
  \defaultfontfeatures{Ligatures=TeX,Scale=MatchLowercase}
\fi
% use upquote if available, for straight quotes in verbatim environments
\IfFileExists{upquote.sty}{\usepackage{upquote}}{}
% use microtype if available
\IfFileExists{microtype.sty}{%
\usepackage[]{microtype}
\UseMicrotypeSet[protrusion]{basicmath} % disable protrusion for tt fonts
}{}
\usepackage{hyperref}
\hypersetup{
            pdfborder={0 0 0},
            breaklinks=true}
\urlstyle{same}  % don't use monospace font for urls
\usepackage[margin=1in]{geometry}
\usepackage{graphicx,grffile}
\makeatletter
\def\maxwidth{\ifdim\Gin@nat@width>\linewidth\linewidth\else\Gin@nat@width\fi}
\def\maxheight{\ifdim\Gin@nat@height>\textheight\textheight\else\Gin@nat@height\fi}
\makeatother
% Scale images if necessary, so that they will not overflow the page
% margins by default, and it is still possible to overwrite the defaults
% using explicit options in \includegraphics[width, height, ...]{}
\setkeys{Gin}{width=\maxwidth,height=\maxheight,keepaspectratio}
\setlength{\emergencystretch}{3em}  % prevent overfull lines
\providecommand{\tightlist}{%
  \setlength{\itemsep}{0pt}\setlength{\parskip}{0pt}}
\setcounter{secnumdepth}{0}
% Redefines (sub)paragraphs to behave more like sections
\ifx\paragraph\undefined\else
\let\oldparagraph\paragraph
\renewcommand{\paragraph}[1]{\oldparagraph{#1}\mbox{}}
\fi
\ifx\subparagraph\undefined\else
\let\oldsubparagraph\subparagraph
\renewcommand{\subparagraph}[1]{\oldsubparagraph{#1}\mbox{}}
\fi

% set default figure placement to htbp
\makeatletter
\def\fps@figure{htbp}
\makeatother

\usepackage{float}
\usepackage[boxruled,vlined]{algorithm2e}
\usepackage{listings}
\usepackage{xcolor}
\usepackage {tikz}
\usepackage{indentfirst}
\usepackage{tabularx}
\usepackage{multirow}
\usepackage{pgfplots}
\usepackage{wrapfig}
\usepackage{booktabs,makecell}
%\usepackage[bottom]{footmisc}
\colorlet{mygray}{black!30}
\colorlet{mygreen}{green!60!black}
\colorlet{mymauve}{red!90}

\lstset{
  tabsize=2,
  backgroundcolor=\color{gray!10},  
  basicstyle=\ttfamily,
  columns=fullflexible,
  breakatwhitespace=false,      
  breaklines=true,                
  captionpos=b,                    
  commentstyle=\color{mygreen}, 
  extendedchars=true,              
  frame=single,                   
  keepspaces=true,             
  keywordstyle=\bfseries\color{blue},      
  language=c++,                 
  numbers=left,                
  numbersep=5pt,
  breaklines=true,
  numberstyle=\tiny, 
  rulecolor=\color{mygray},        
  showspaces=false,               
  showtabs=true,                                  
  stringstyle=\color{mymauve},                          
  title=\lstname                
}

\definecolor{light-gray}{gray}{0.9}
\newcommand{\code}[1]{\colorbox{light-gray}{\texttt{#1}}}
\newcommand{\pnt}[1]{{\scriptstyle#1}}
\let\origfigure\figure
\let\endorigfigure\endfigure
\renewenvironment{figure}[1][2] {
    \expandafter\origfigure\expandafter[H]
} {
    \endorigfigure
}
\usepackage{etoolbox}
\makeatletter
\providecommand{\subtitle}[1]{% add subtitle to \maketitle
  \apptocmd{\@title}{\par {\large #1 \par}}{}{}
}
\makeatother
\ifnum 0\ifxetex 1\fi\ifluatex 1\fi=0 % if pdftex
  \usepackage[shorthands=off,main=english]{babel}
\else
  % load polyglossia as late as possible as it *could* call bidi if RTL lang (e.g. Hebrew or Arabic)
  \usepackage{polyglossia}
  \setmainlanguage[]{english}
\fi

\title{\textbf{Project Report}\\
\Large{Implementations of Finger Search:}}
\providecommand{\subtitle}[1]{}
\subtitle{An Extended Feature of Some Data Structures}
\author{Minh Thang Cao}
\date{23 March 2021}

\begin{document}
\maketitle

\graphicspath{ {./} }

\hypertarget{section1}{%
\section{\texorpdfstring{1.
\enspace Introduction}{1. Introduction}}\label{section1}}

Finger Search is an extended feature, which is suitable for some popular
data structures, that improves the running time of search operations as
well as other ones that depend on searching. Finger Search uses a
pointer to an element, called \emph{finger}, in the data structure to
reduce the length of search path. Instead of searching from a normal
start point, Finger Search starts at the \emph{finger}. Let \(d\) be the
difference between ranks\footnote{Denoted \(i\) if \(x\) is the
  \(i^{th}\) smallest element in the data structure (consistent with
  only smallest or largest for all elements)} of the \emph{finger} and
the target value, the running time of Finger Search is proportional to
\(d\).

\hypertarget{section2}{%
\section{\texorpdfstring{2. \enspace Finger Search on
Treaps}{2. Finger Search on Treaps}}\label{section2}}

\hypertarget{overview}{%
\subsection{2.1. Overview}\label{overview}}

Briefly, Treap is a Randomized Binary Search Tree whose properties are
the combination of Binary Search Tree and Heap. Each nodes in Treap
holds a randomized priority which is used to maintain the Heap
properties. The nodes order in a Treap satisfies ascending values in an
inorder traversal while their priority satisfies min-heap (or max-heap)
property, in which priority of a node is always smaller (or larger) than
its children. This randomized combination creates a special property
that the \emph{expected} height of a Treap is \(O(\log n)\) in average,
with \(n\) is the total number of nodes. Therefore, it supports
searching and some associated operations in expected \(O(\log n)\) time
which is standard for most balanced trees. Since Finger Search achieve
its goal in an expected running time, Treap is a very suitable
randomized data structure for Finger Search.

In this section, Finger Search on Treaps will be introduced along with
my implementation. Due to R. Seidel and C.R. Aragon (see \cite{1}),
there are several ways to achieve the running time proportional to the
distance \(d\) such as using multiple pointers, multiple keys, or even
without anything by relying on the shortness of the search path, and
these approaches all give us the \(O(\log d)\) expected running time,
see \cite{1} for more details. What makes Finger Search different from
normal search on Treaps is that if we start at the finger \(f\) and look
for a node \(v\), Finger Search would points out the lowest common
ancestor, denoted LCA, of \(f\) and \(v\), then it performs searching
down as normal. Since if we do Finger Search without any assistance of
other elements, the excess path in some cases may be longer compared to
the path that we are supposed to search on. To minimize the probability
that we get these cases, multiple extra pointers are definitely useful,
which is the approach used in this implementation.

\begin{figure}
\centering
\vspace{1mm}
\includegraphics[width=0.4\textwidth]{tree.png}
\caption{\label{fig1:figs}Visualization of \textit{left parent} and \textit{right parent} of a node $u$ in a binary tree.}

\  
\hrule
\end{figure}

Based on what R. Seidel and C.R. Aragon found (see Section 5.4 of
\cite{1}), denote a \emph{left parent} or a \emph{right parent} of a
node \(u\) is the first ancestor of \(u\) on the path from \(u\) to the
\emph{root} node so that \(u\) is on the ancestor's right or left
subtree, respectively (see Figure \ref{fig1:figs}). With this property,
we can get the LCA quickly by jumping through \emph{left parents} or
\emph{right parents} without traveling through all nodes on the upward
path. From the LCA, we start doing Binary Search downwards to get to the
node \(v\).

\hypertarget{implementation}{%
\subsection{2.2. Implementation}\label{implementation}}

Since Finger Search does not affect much on the normal implementation of
Treap, only some small changes are made outside of the Finger Search
function. The main addition is a new \emph{finger} node pointer property
of the Treap class. Other important changes are the Node class's
properties and rotation functions.

\hypertarget{the-node-class}{%
\subsubsection{2.2.1. The Node class}\label{the-node-class}}

To handle \emph{left parent} and \emph{right parent} which vary between
different nodes, I added two corresponding pointers to each node other
than the actual parent. Other properties remain the same. The code is
given below. \newpage

\begin{lstlisting}
template<typename T>
class Node {
  public:
    T data;
    double priority;
    Node *parent, *left, *right;
    Node *leftParent, *rightParent; // new properties
    Node(double value, Node* parent=nullptr,
         Node* left=nullptr, Node* right=nullptr,
         Node* leftParent=nullptr, Node* rightParent= nullptr) {
      this->data = value;
      this->parent = parent;
      this->left = left;
      this->right = right;
      this->leftParent = leftParent;
      this->rightParent = rightParent;
      priority = Random::getReal(0, 1);
    };
};
\end{lstlisting}
\vspace{-7mm}

\hypertarget{finger-search}{%
\subsubsection{2.2.3. Finger Search}\label{finger-search}}

As described above, Finger Search would find the LCA of the finger and
the target node before doing normal Binary Search from the LCA. Let
assume \(f\) and \(v\) be the value in the finger and the target value.
The procedure is described below:

\begin{enumerate}
\def\labelenumi{\arabic{enumi}.}
\tightlist
\item
  Check if \(f = v\), then return the \emph{finger}.
\item
  If \(f > v\), we start chasing \emph{left parent} from \emph{f} until
  reaching a null pointer or a node has data greater than both \emph{f}
  and \emph{v}. The previous node at the end of the loop is the LCA. If
  \(f < v\), we chase \emph{right parent} instead.
\item
  Perform the usual downward search from the LCA.
\item
  Before returning the node, if it is found, it will be the new
  \emph{finger}.
\end{enumerate}

The C++ code is provided below.

\begin{lstlisting}
template<typename T>
Node<T>* Treap<T>::fingerSearch(T value) {
  if (value == finger->data) return finger;
  Node<T>* LCA = root; // The lowest common ancestor
  Node<T>* current = finger;
  if (value > finger->data) {
    while (current && current->data <= value) {
      LCA = current;
      current = current->rightParent;
    }
  } else {
    while (current && current->data >= value) {
      LCA = current;
      current = current->leftParent;
    }
  }
  Node<T>* foundNode = binarySearch(value, LCA); // downard search from LCA
  if (foundNode) {
    finger = foundNode;
  }
  return foundNode;
}
\end{lstlisting}

\vspace{-7mm}

\hypertarget{the-rotation-functions}{%
\subsubsection{2.2.2. The rotation
functions}\label{the-rotation-functions}}

Rotation is a main part of Treaps which maintains the heap property.
Since a node's parent may be changed after a rotation, we need to modify
the corresponding functions to keep the nodes' \emph{left} and
\emph{right parents} correct. The changes are a very simple for both
left and right rotation. For the right rotation at a node \(u\), the
only two node has their ``special parents'' changed are \(u\) and the
left child of \(u\), see Figure \ref{fig2:figs} for example. Even though
the right child of \(u\)'s left child changes its parent, which is \(u\)
after the rotation, its \emph{left parent} and \emph{right parent} do
not change. The procedure is completely identical for the left rotation.

\begin{figure}
\centering
\vspace{1mm}
\includegraphics[height=0.3\textwidth]{TreeRotation.png}
\caption{\label{fig2:figs}Visualization of right rotation at node 4. The colored nodes are ones has their \textit{left parent} or \textit{right parent} changed. The \textit{right parent} of node 2, the left child of 4, changed to the \textit{right parent} of 4, and \textit{left parent} of node 4 changed to 2. Left rotation is identical.}

\  
\hrule
\end{figure}

Going to the real life implementation, those changes are only a couple
of lines for each rotation type. The implementations of left and right
rotation are given below.

\begin{lstlisting}
template<class T>
void Treap<T>::leftRotate(Node<T>* &node) {
  Node<T> *parent = node->parent;
  Node<T> *r = node->right;
  node->parent = r;
  node->rightParent = r; // set right parent of the given node
  r->leftParent = node->leftParent; // set left parent of right child 
  if (r->left) {
    r->left->parent = node;
  }
  node->right = r->left;
  r->left = node;
  r->parent = parent;
  if (parent) {
    if (parent->left == node) {
      parent->left = r;
    } else {
      parent->right = r;
    }
  }
  node = r;
}

template<class T>
void Treap<T>::rightRotate(Node<T>* &node) {
  Node<T> *parent = node->parent;
  Node<T> *l = node->left;
  node->parent = l;
  node->leftParent = l; // set left parent of the given node
  l->rightParent = node->rightParent; // set right parent of left child 
  if (l->right) {
    l->right->parent = node;
 }
  node->left = l->right;
  l->right = node;
  l->parent = parent;
  if (parent) {
    if (parent->left == node) {
      parent->left = l;
    } else {
      parent->right = l;
    }
  }
  node = l;
}
\end{lstlisting}

\hypertarget{running-time}{%
\subsection{2.3. Running time}\label{running-time}}

Let \(d\) be a difference between \emph{ranks}\footnote{position of the
  value in the sorted array of all values on the tree} of any two node
values on a Treap. Due to Seidel and Aragon \cite{1}, the expected
running time of Finger Search on Treap theoretically is \(O(\log d)\).
Therefore, the closer to the \(finger\) the target value is, the faster
the expected running time will be. Thus, to test and prove if the
running time of Finger Search in practice is identical to the theory,
the search value would be chosen based on a given \(d\).

At first, the Treap is initialized with \(5\,000\,000\) nodes that
contain all values from 1 to \(5\,000\,000\), so it is easy to see that
the difference between \(ranks\) of two nodes are exactly the difference
between their values. The chosen values of \(d\) for testing starts from
a small value (20) and is doubled when \(d\) increases, specifically let
\(D = [20, 20\cdot2^1, 20\cdot2^2, \ldots, 20\cdot2^{15}]\). The testing
program will do the following procedure: \vspace{-4mm}

\begin{enumerate}
\def\labelenumi{\arabic{enumi}.}
\tightlist
\item
  Picks a random node in the Treap and set it to be the \(finger\).
\item
  For each \(d\) in \(D\): \vspace{-4mm}
\end{enumerate}

\begin{quote}
\begin{itemize}
\tightlist
\item
  Repeat \(5\,000\,000\) times: search for a value \(finger + d\) or
  \(finger - d\) which is chosen randomly in the two. Each time a node
  is found, the \(finger\) is set to that node.
\item
  Calculate the average running time of the trials.
\end{itemize}
\end{quote}

See figure \ref{fig:TreapResult} below for the result data.

\begin{figure}
\centering
\begin{minipage}{1\textwidth}
  \centering
    \begin{tabularx}{0.8\textwidth}{|>{\centering\arraybackslash}X|>{\centering\arraybackslash}X|>{\centering\arraybackslash}X|}
  \hline
      $d$  & Finger Search time  & Binary Search form root time \\ \hline
     20  & 7.89250  & 27.2041 \\ \hline
     40  & 9.95326  & 28.0649 \\ \hline
     80  & 11.9180  & 28.8736 \\ \hline
    160  & 13.9715  & 27.4262 \\ \hline
    320  & 15.8764  & 28.0812 \\ \hline
    640  & 18.3024  & 29.7978 \\ \hline
   1280  & 20.2644  & 29.4003 \\ \hline
   2560  & 22.3453  & 28.0397 \\ \hline
   5120  & 24.6583  & 28.8494 \\ \hline
  10240  & 26.4185  & 28.1270 \\ \hline
  20480  & 28.3653  & 28.4053 \\ \hline
  40960  & 30.1493  & 28.3681 \\ \hline
  81920  & 32.8924  & 28.7313 \\ \hline
 163840  & 33.6583  & 27.9549 \\ \hline
 327680  & 35.6414  & 27.9997 \\ \hline
 655360  & 40.2822  & 27.4228 \\ \hline
  \end{tabularx}
\end{minipage}
\caption[Caption]{Average running time of Finger Search and Binary Search from root on Treap with $d$ in $[20, 20\cdot2^1, 20\cdot2^2, \ldots, 20\cdot2^{15}]$. The total number of search times for each $d$ is $5\,000\,000$. The running times are in \textit{number of nodes visited} at the end of the search function.}
\label{fig:TreapResult}
\end{figure}

\newpage

As you can see, while the running time of Finger Search increases when
\(d\) increases, the usual Binary Search from root remains about the
same for all \(d\), so we can say that Finger Search depends on \(d\).
Obviously, since the time increases about 2 when \(d\) is doubled, see
figure \ref{fig:TreapResult}, the running time of Finger Search is
approximately \(2\log d = O(\log d)\). From that, in the below graph,
the running time for each \(d\) in the graph is divided by \(2\log d\),
denoted \(r_d\) ratio. To see how the running time of Finger Search
related to \(O(\log d)\) in practice, figure \ref{fig:TreapGraph} is
provided below. It is clear that the ratios calculated are almost the
same for all \(d\). This reflects that the running time of Finger Search
is \(O(\log d)\) in practice.

\begin{figure}
\begin{minipage}{0.95\textwidth}
\begin{center}
\begin{tikzpicture}
  \begin{axis}[ 
    symbolic x coords={0, 20, 40, 80, 160, 320, 640, 1280, 2560, 5120, 10240, 20480, 40960, 81920, 163840, 327680, 655360, 1310720},
    xtick={0, 20, 40, 80, 160, 320, 640, 1280, 2560, 5120, 10240, 20480, 40960, 81920, 163840, 327680, 655360, 1310720},
    xticklabels={0,$2^0$,$2^1$, $2^2$,$2^3$,$2^4$,$2^5$,$2^6$,$2^7$,$2^8$,$2^9$,$2^{10}$,$2^{11}$,$2^{12}$,$2^{13}$,$2^{14}$,$2^{15}$,$2^{16}$},
    scaled x ticks={real:20},
    height=5cm,
        width=14cm,
    minor tick num=5,
    grid=both,
    grid style={line width=.1pt, draw=gray!20},
    major grid style={line width=.2pt,draw=gray!40},
    ymin=0,
        ymax=2,
    xmin=0,
        xmax=1310720,
    xlabel=$d$,
    ylabel={$r_d$},
        ylabel style={rotate=-90}
  ] 
        \addplot file[skip first] {TreapGraph.txt};
  \end{axis}
\end{tikzpicture}
\end{center}
\end{minipage}
\caption[Caption]{The graph of ratios $r_d$ versus different values of $d$ given in figure \ref{fig:TreapResult}.}
\label{fig:TreapGraph}
\end{figure}
\hrule

\hypertarget{section3}{%
\section{\texorpdfstring{3. \enspace Finger Search on
Skiplists}{3. Finger Search on Skiplists}}\label{section3}}

\newpage
\begin{thebibliography}{9}
\bibitem{1}
Seidel, R., Aragon, C.R. Randomized search trees. \emph{Algorithmica} 16, 464–497 (1996). \url{https://doi.org/10.1007/BF01940876}

\end{thebibliography}

\end{document}
